The basic task for any probabilistic system is to compute the conditional probability distribution for a set of \textbf{query variables}, given some \textbf{evidences}. By the way,
we will consider only one query variable at the time, generally named \textbf{simple query}, that looks like:
\begin{center}
    $\mathbf{P}(X_i|E = e)$
\end{center}
where
\begin{itemize}
    \renewcommand{\labelitemi}{-}
    \item $\mathbf{P}$ is the probability distribution.
    \item $X_i$ defines the query variable, or more simply what we are looking for.
    \item $E = e$ denotes the set of evidence variables, and e is a particular observed event that belongs to one of the random variables.
\end{itemize} \vspace{3.5pt}

We have already seen how to compute posterior probabilities, such as in the burglary network. From the same network, we might ask which is the probability that a 
burglary occured given $JohnCalls$ and $MaryCalls$ true:
\begin{center}
    $\mathbf{P}(Burglary|JohnCalls=True, MaryCalls=True) = \langle 0.284, 0.716 \rangle$
\end{center}
In this section we discuss a smart way for computing this conditional probability without constructing its explict representation.