The previous chapter noted the importance of absolute and conditional independence relationships in simplifying probabilistic representation.
This section introduces a systematic way to represent such relationships in the form of \textbf{Bayesian networks}.
We define the syntax and semantics of these networks and show how they can be used to capture uncertain knowledge. \vspace{7pt}

A Bayesian network is a simple graphical notation for conditional independence assertions and hence for a compact specification of full joint distribution.
The Bayesian network's syntax is composed by:
\begin{enumerate}
    \item Each node corresponds to a random variable.
    \item A set of directed links or arrows connects pairs of nodes.
    \item Each node $X_i$ has a conditional probability distribution $\mathbf{P}(X_i|Parents(X_i))$, that quantifies the effect of the parents on the node.
\end{enumerate}
\begin{example}
    i.e. Topology of network encodes conditional independence assertions:
    \begin{center}
        % \includegraphics{16/50}
    \end{center}
    \begin{itemize}
        \renewcommand{\labelitemi}{-}
        \item Weather is independent of the other variables \ownfootnote{Formally, the absolute or conditional independence is indicated by the absence of a link between nodes.}.
        \item Toothache and Catch are conditionally independent given Cavity \ownfootnote{The intuitive meaning of an arrow is typically that X has a direct influnce on Y, which suggests that causes should be parents of effects.}.
    \end{itemize}
\end{example}
\begin{example}
    i.e. I'm at work, neighbor John calls to say my alarm is ringing, but neighbor Mary doesn't call. Sometimes it's set off by minor earthquakes. Is there a burglar? \vspace{3.5pt}

    The random variables are: \it Burglar, Earthquake, Alarm, MaryCalls, JohnCalls.
    \begin{center}
        % \includegraphics{18/50}
        \ownfootnote{The network topology reflects \textbf{causal} knowledge, from the causes nodes we define the effects nodes.}
        \ownfootnote{For each node the conditional distribution are shown as a \textbf{conditional probability table}, or simply CPT.}
        \ownfootnote{Let's take a look at the tables. In this network we are talking about joint distribution, not full joint distribution. Simply, the full joint distribution about boolean random variables can be computed by $1-P(a)$.}
    \end{center}
\end{example}