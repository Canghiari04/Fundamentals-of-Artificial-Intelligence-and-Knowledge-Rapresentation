The exponential growth in the number of game states appears to be the biggest problem with the Min-Max algorithm. We have discovered that computers simply play all the possible
matches, evaluate leaf nodes and then propagate back the initial choice. Additionally, Min-Max revised algorithm shows us that the resulting game tree generates soo many useless moves,
even though they will never occur in a real scenario. \vspace{3.5pt}

Therefore, the following idea to avoid this bad behavior may be to reduce the size of the tree. The best-known technique is \textbf{Alpha-Beta Pruning}. \vspace{3.5pt}

Consider the two players game used before. \vspace{3.5pt}
\begin{center}
    \includegraphics[width=0.6\textwidth]{img/img14.png}
\end{center} \vspace{3.5pt}

The outcome received by the figure is that we can identify the minmax choice without evaluate the other two final states of the node $C$. Another way to express what happens
can be described through a semplification of Min-Max formula. \vspace{3.5pt}

Let the two unevaluated states of node $C$ have values x and y. Then the utility value of the root node is given by:
\begin{itemize}[nosep]
    \renewcommand{\labelitemi}{}
    \item $minmax(root) = max(min(3,12,8), min(2,x,y), min(15, 5, 2))$ 
    \item $minmax(root) = max(3, 2, 2)$
    \item $minmax(root) = 3$.
\end{itemize} \vspace{3.5pt}

In other words, it doesn't matter which are the real values of the unvaluated expressions, MAX will choose always the utility value $3$, such that is the highest number between
all the possible choices. Starting from some definitions we can generalize the algorithm behavior. 
\begin{definition}
    The values in \textbf{MAX} nodes are \textbf{Alpha-values}.
\end{definition}
\begin{definition}
    The values in \textbf{MIN} nodes are \textbf{Beta-values}.
\end{definition}

Given these assumptions, we summarize Alpha-Beta Pruning method, splitting the cases in base of which player is currently moving. 
\begin{definition}
    The game tree is pruned as follows:
    \begin{itemize}[nosep]
        \renewcommand{\labelitemi}{-}
        \item If a \textbf{Alpha-value} is greater than a \textbf{Beta-value} of a descending node: stop the generation of children of the descending node.
        \item If a \textbf{Beta-value} is smaller than a \textbf{Alpha-value} of a descending node: stop the generation of children of the descending node.
    \end{itemize}
\end{definition}

The effectiveness of Alpha-Beta Pruning is highly dependent on the order in which the states are examined. For instance, we could not prune any successors of node $D$ at all
because the worst successors were generated first. If the third successor of $D$ had been generated first, we would have been able to prune the other two. If this can be done\footnote{When the best nodes are evaluated first.},
then it turns out that Alpha-Beta requires $O(b^{m/2})$ to pick the best move, instead of $O(b^m)$ for Min-Max. Anyway, in the average case with random distribution of the node
values, the number of nodes examined becomes about $O(b^{3/4})$.