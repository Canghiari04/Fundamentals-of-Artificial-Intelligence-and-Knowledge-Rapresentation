\textbf{Hill-climbing search} is a very basic local search algorithm. It is simply a loop that continually moves in the direction of the increasing value, choosing always
the highest solution inside the neighborhood. It terminates when it reaches a \textbf{peak}, that might be a local or global maximum, and there isn't a higher value 
inside the set of neighbors of the current solution. \vspace{3.5pt}

Despite its simplicity, this method often gets stuck for the following reasons:
\begin{itemize}
    \renewcommand{\labelitemi}{-}
    \item \textbf{Local maxima}: a local maximum is a peak that is higher than each of its neighboring states but lower than the global maximum. So, given $s^*$ a local peak
    and $f$ an evaluation function, such that for any state $s$ that belongs to the neighborhood $N(s^*)$, we have: \vspace{3.5pt}
    \begin{center}
        $f(s^*) \ge f(s)$ $\forall s \in N(s^*)$.
    \end{center}
    \item \textbf{Ridges}: a ridge is a sequence of local maxima, that are very difficult for hill-climbing to reach. When we sample such a function, the sampling points
    will not fall exactly on the ridge line and will float nearby.
    \item \textbf{Plateaux}: a plateau is flat area of the landscape, where the neighboring states have the same value of the current solution.
\end{itemize}

Moreover, hill-climbing search cannot be used for complex problems; first of all, it has a very local view and, in the other hand, forgets about the solution already visited.
It's like: \vspace{3.5pt}
\begin{center}
    \textit{Climbing Everest in thick fog with amnesia}.
\end{center}\vspace{3.5pt}