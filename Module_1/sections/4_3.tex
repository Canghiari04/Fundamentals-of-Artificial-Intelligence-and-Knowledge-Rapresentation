A \textbf{genetic algorithm} is a search algorithm in which successor states are generated by picking two parent states, instead of choosing one state from the neighborhood of
the current solution. Before moving on, we define some common terms about genetic algorithms. \vspace{3.5pt}

Genetic algorithms are inspired by the process of \textbf{evolution}, composed as follows:
\begin{itemize}
    \renewcommand{\labelitemi}{-}
    \item \textbf{Inheritance}: offspring resemble their parents.
    \item \textbf{Adaptation}: organisms are suited to their habitats.
    \item \textbf{Natural selection}: new types of organisms emerge and those that fail to change are subject to extinction.
\end{itemize}

It's important to note that the specific terminology for genetic algorithms differs from that used for local search algorithms, it becomes:
\begin{itemize}
    \renewcommand{\labelitemi}{-}
    \item Each single solution defines a \textbf{genotype}.
    \item The set of solutions is called \textbf{population}.
    \item Chromosomes are made of units called \textbf{genes}.
    \item The domain of values of a gene is composed of \textbf{alleles}.
\end{itemize} \vspace{3.5pt}

Described genetic algorithms' terminology, let's examine how they work. Genetic algorithms begin with a set of $k$ randomly generated states, which represents the
\textbf{population}. \dots