A \textbf{genetic algorithm} is a search algorithm in which successor states are generated by picking two parent states, instead of choosing one state from the neighborhood of
the current solution. Before moving on, we define some common terms about genetic algorithms. \vspace{3.5pt}

Genetic algorithms are inspired by the process of \textbf{evolution}, composed as follows:
\begin{itemize}
    \renewcommand{\labelitemi}{-}
    \item \textbf{Inheritance}: offspring resemble their parents.
    \item \textbf{Adaptation}: organisms are suited to their habitats.
    \item \textbf{Natural selection}: new types of organisms emerge and those that fail to change are subject to extinction.
\end{itemize}

It's important to note that the specific terminology for genetic algorithms differs from that used for local search algorithms, it becomes:
\begin{itemize}
    \renewcommand{\labelitemi}{-}
    \item Each single solution defines a \textbf{genotype}.
    \item The set of solutions is called \textbf{population}.
    \item Chromosomes are made of units called \textbf{genes}.
    \item The domain of values of a gene is composed of \textbf{alleles}.
\end{itemize} \vspace{3.5pt}

Having described the terminology of genetic algorithms, let's examine how they work. Genetic algorithms begin with a set of $k$ randomly generated states, representing the
\textbf{population}. Each state, or \textbf{genotype}, is represented as a string over a finite alphabet, most commonly, a string of 0s and 1s. For example, in the image
below, which shows the 8-queens problem, each state can be specified as the positions of the 8 queens. Therefore, each genotype is an array of eight elements and the current 
position assumed by the queen will be inserted within each cell. \vspace{3.5pt}
\begin{center}
    \includegraphics[width=0.3\textwidth]{img/img6.png}
\end{center} \vspace{3.5pt}

Defined the representation of states, each of them is rated by a \textbf{fitness function}, such that the algorithm will take some genotypes for the production of the next 
generation of states. A fitness function should return higher values for better states, so, for the 8-queens problem we use the number of non-attacking pairs of queens. In 
this case, the probability of being choosen for reproducing is directly proportional to the \textbf{fitness score}. \vspace{3.5pt}

According to the fitness probabilities, pairs of genotypes are taken from the population and then combined for the reproduction. For each pair, a \textbf{crossover point} is 
chosen randomly from the position in the string. As in the example introduced, portions of the array representation are swapped. \vspace{3.5pt}

Finally, the offspring generated is subject to a random \textbf{mutation}, one of the digit that composed the array representation is changed; this corresponds to choosing a 
queen randomly and moving it to a random cell. A visual summary on the main steps of genetic algorithms is shown below. \vspace{3.5pt}
\begin{center}
    \includegraphics[width=0.9\textwidth]{img/img7.png}
\end{center} \vspace{3.5pt}