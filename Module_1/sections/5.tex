Concerning population based methods, now we explore another type of algorithms known as \textbf{Swarm Intelligence} algorithms.
\begin{definition}
    \textbf{Swarm Intelligence}, briefly SI, is an artificial intelligence technique based around the study of collective behavior in decentralized and self-organized systems.
\end{definition}

A simpler definition expresses Swarm Intelligence algorithms as a set of artificial intelligence methods that rely on \textbf{AI agents}, where each of them is totally autonomus
and independent. So, any agent has its own goals and decision algorithm. \vspace{3.5pt}

The entire swarm, in any kind of algorithm, has this nice feature: we can always examine an emerging behavior that comes from the interactions taken by any single agent\footnote{The \textbf{emerging behavior} does not come from the planning established by the algorithm!}.
Therefore, the key idea about Swarm Intelligence is to take some individuals, that are simple entities, and put them all together, thus defining a quite complex self-organization. \vspace{3.5pt}

These agents can communicate either in a \textbf{direct} and \textbf{indirect} way. The last one way is performed by \textbf{stimmergy}, in other words, changing the environment 
they can communicate each other. \vspace{3.5pt}

These methods are:
\begin{itemize}[nosep]
    \renewcommand{\labelitemi}{-}
    \item \textbf{Adaptive}, because they adapt quickly to the changes in the environment.
    \item \textbf{Robust}, because if some agents stop working, the rest of the swarm still doing its job.  
\end{itemize}

Each type of self-organization is based on three major ingredients, as follows:
\begin{itemize}[nosep]
    \renewcommand{\labelitemi}{-}
    \item \textbf{Multiple interactions among agents}. \\ As we said, the organization is composed by some simple agent, and each of them has its own goals and decision algotithm. Putting them together, we are defining a \textbf{multi-agent system}: from the taken interactions we can retrieve an emerging behavior.
    \item \textbf{Positive feedback}. \\ Imitating, with positive feedback, successfull behaviors. 
    \item \textbf{Negative feedback}. \\ Suppose we have a \textit{food source}, for us means a solution with a higher fitness score. If the food is already exhausted, so in the neighborhood of the current solution $N(s)$ nothing is left, we move to somewhere else.
\end{itemize}

Swarm Intelligence algorithms are driven by natural phenomena, in particular, by groups of animals that interacting allow us to extrapolate an emerging behavior. We will consider
three main algorithms, which are:
\begin{itemize}[nosep]
    \renewcommand{\labelitemi}{-}
    \item Ant Colony Optimization.
    \item Particle Swarm Optimization.
    \item Artificial Bee Colony Algorithm.
\end{itemize}