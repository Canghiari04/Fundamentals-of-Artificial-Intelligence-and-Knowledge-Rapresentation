\textbf{Iterative deepening search} is a strategy usually used with the depth-first search algorithm, which defines iteratevely the \textit{best depth limit}. It does this by 
gradually increasing the limit $l$, as we have seen in depth-limited search, until the target node is found. This will occur when the depth limit $l$ reaches $d$, the depth of the
shallowest node. \vspace{3.5pt}

It combines the advantages of depth-first search and breadth-first search strategies. Like DFS, its memory complexity is linear, to be precise $O(bd)$\footnote{Iterative deepening search, such as the depth-first search, only maintains in memory one path at the time; it is discarded if it does not contains the goal node.}. 
Like BFS, it is \textbf{complete} when the brancing factor $b$ is finite and \textbf{optimal} when the path cost increases as a function of the node depth. \vspace{3.5pt}

Iterative deepening may seem like a waste of computation because the initial nodes are generated multiple times, but the execution time never gets worsen. The reason 
is that in a search tree with the same branching factor at each level, most of the nodes are in the bottom level, so it does not matter much that the upper levels are generated
multiple times. Let's consider an example for better understanding.
\begin{example}
    i.e. Considering the branching factor equal to 10 and the depth of the shallowest node equal to 5, briefly: \vspace{3.5pt}
    \begin{center}
        \begin{itemize}
            \renewcommand{\labelitemi}{}
            \item $b = 10$
            \item $d = 5$
        \end{itemize}
    \end{center} \vspace{3.5pt}

    and given the equation which defines the total number of nodes generated by iterative deepening in the worst case \vspace{3.5pt}
    \begin{center}
        $N(IDS) = (d)b + (d - 1)b^2 + \dots + (1)b^d$        
    \end{center} \vspace{3.5pt}

    we have \vspace{3.5pt}
    \begin{center}
        $N(IDS) = 50 + 400 + 3.000 + 20.000 + 100.000 = 123.450$        
    \end{center} \vspace{3.5pt}

    while we could get the same amount if we decide to use breadth-first search strategy \vspace{3.5pt}
    \begin{center}
        $N(BFS) = 10 + 100 + 1.000 + 10.000 + 100.000 = 111.110$        
    \end{center} \vspace{3.5pt}
\end{example}