Many AI problems can be solved by exploring the \textbf{solution space}. A solution space is a set of all the sequences of actions that an agent can apply. The agent examines all
the possible sequences of actions and chooses the best one. The goal is to reach the solution starting from a \textbf{initial state}. The process of trying different sequences
is called \textbf{search}. \vspace{3.5pt}

Usually, it's useful to think about the \textbf{search} process as a \textbf{search tree}, where:
\begin{itemize}
    \renewcommand{\labelitemi}{-}
    \item The initial state corresponds to the \textbf{root} of the tree.
    \item Each branch that makes up the tree defines the \textbf{action} that can performed by the current node.
    \item The nodes represent the subsequent reachable states. However, a certain node can be a \textbf{leaf node}. A leaf node is a new state to expand, a solution or a dead-end.
\end{itemize}
Previously, we said that a solution is a sequence of actions, so we need to define two main operations that allow us to build a sequence. We do this by \textbf{expanding}
the current state, applying each possible action to the current node, \textbf{generating} a new set of states\footnote{Every time we expand the current node, new state are generated.}.
Generally, the set of all leaf nodes available for expansion at any given point is called \textbf{frontier} or \textbf{fringe}. The process of expanding nods continues until
either a solution is found or there are no more states to expand. \vspace{3.5pt}

Finally, concluding this first introduction, we say that all the search algorithms are named \textbf{search strategies}, and typically they all share the same structure, varying
by the way they choose which state needs to be expand.