Before discussing the large set of search strategies, we need to consider the basic criteria that define the effectiveness of the algorithms. Typically, the performance of an algorithm
is evaluated by four ways, as follows:
\begin{itemize}
    \renewcommand{\labelitemi}{-}
    \item \textbf{Completeness}: does the algorithm guarantees to find a final solution?
    \item \textbf{Optimality}: does the strategy find the best solution?
    \item \textbf{Time complexity}: how long does the algorithm take to find a solution?
    \item \textbf{Space complexity}: how much memory is needed to carry out the search?
\end{itemize}
As we said, search algorithms are called search strategies, or briefly \textbf{strategies}. Strategies are divided into two main types:
\begin{itemize}
    \renewcommand{\labelitemi}{-}
    \item \textbf{Non-informed strategies}: they don't use any domain knowledge, apply rules arbitrarily, and do exhaustive search.
    \item \textbf{Informed strategies}: they use domain knowledge, apply rules following \textbf{heuristics}.
\end{itemize}