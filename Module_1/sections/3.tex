This paragraph shows how \textbf{informed search strategies} can find solutions more efficiently than \textbf{non-informed search strategies}.

The general approach about informed search strategies is defined by \textbf{best-first search strategy}. Best-first algorithm select a node for the expansion based on an 
\textbf{evaluation function}, briefly written as $f(n)$. Generally, this type of search strategy uses evaluation function as an estimation of the effort required to reach the 
final state; the node with the lowest evaluation is expanded first. \vspace{3.5pt}

The choice of the evaluation function $f(n)$ determines the search strategy. Most best-first algorithms include as a component of $f(n)$ a \textbf{heuristic function}, denoted
$h(n)$. \vspace{3.5pt}
\begin{center}
    $h(n) =$ estimated cost of the path from the current node $n$ to the final state\footnote{Until now, we saw the $g(n)$ function as the path cost from the root node to the
    current state; selecting the node for the expansion with the lowest $g(n)$. Instead the heuristic function $h(n)$ defines the distance from the current node to the final state!}.
\end{center}
We will examine two main applications of the best-first search strategy, which are:
\begin{itemize}
    \renewcommand{\labelitemi}{-}
    \item Greedy best-first search.
    \item A* search.
\end{itemize} 